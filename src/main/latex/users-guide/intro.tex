\chapter{Introduction}

LaBoGrid is a Lattice Boltzmann-based, experimentation-oriented, flow simulation
tool. It can be executed in a wide range of environments: from desktop computers
to distributed environments such as clusters.
LaBoGrid is written in Java and is therefore directly executable on most modern
architectures and in heterogeneous clusters (regarding architecture, installed
software, etc.).

LaBoGrid is well suited to experimentation: it features methods to easily
describe the sequence of high-level simulation elements to execute, without the
need to produce additional source code. In addition, if the available simulation
elements do not meet the user's requirements, the user can easily develop new
simulation elements. LaBoGrid is therefore easy to extend.

LaBoGrid was initially developed in the frame of a research project conducted at
the University of Liège (Belgium). The software is essentially based on
the theory presented in 2 thesises~\cite{Beugre10,Dethier11}.

This document describes LaBoGrid, its features (see chapter~\ref{sec_feat}) and,
very briefly, its architecture (see chapter~\ref{sec_arch}). It also shows how
LaBoGrid can be configured (see chapter~\ref{sec_conf}) and executed (see
chapter~\ref{sec_exec}) in order to simulate flows. Finally, the way LaBoGrid
can be extended is presented (see chapter~\ref{sec_ext}) and some additional
tools provided with LaBoGrid are described (see chapter~\ref{sec_add}).
