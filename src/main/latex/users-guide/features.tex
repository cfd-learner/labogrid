\chapter{Features}
\label{sec_feat}

This chapter presents the main features of LaBoGrid. These are described in
following sub-sections.


%%%%%%%%%%%%%%%%%%%%%%%%%%%%%%%%%%%%%%%%%%%%%%%%%%%%%%%%%%%%%%%%%%%%%%%%%%%%%%%%
\section{Configuration}

LaBoGrid's configuration and the simulations it executes are described in a
single configuration file given as input when executing LaBoGrid. See
chapter~\ref{sec_conf} for more details.


%%%%%%%%%%%%%%%%%%%%%%%%%%%%%%%%%%%%%%%%%%%%%%%%%%%%%%%%%%%%%%%%%%%%%%%%%%%%%%%%
\section{Execution environments}

LaBoGrid was designed to be executed on cluster of heterogeneous computers. It
is able to balance efficiently the processing load among available computers in
order to minimize simulations execution time (see sections~\ref{sec_arch_load}
and~\ref{sec_conf_middle_load}). In addition, LaBoGrid features mechanisms that
render it tolerant to failures i.e. unexpected interruptions of some processes
(see sections~\ref{sec_arch_fault} and~\ref{sec_conf_middle_fault}).

It can also be executed on a single computer and efficiently make use of any
number of available cores.


%%%%%%%%%%%%%%%%%%%%%%%%%%%%%%%%%%%%%%%%%%%%%%%%%%%%%%%%%%%%%%%%%%%%%%%%%%%%%%%%
\section{Flow simulation}

LaBoGrid is able to simulate monophasic 3D flows in potentially complex
structures. Lattice Boltzmann methods are used. A D3Q19 lattice (see
section~\ref{sec_conf_lbConf_latt}) represents the state of the flow which is
driven either by pressure conditions, either by a body force applied at each
point of the fluid.

Two collision operators (see section~\ref{sec_conf_proc_op})
are available~: a Single Relaxation Time (SRT) operator\footnote{The SRT
operator is actually the well known Bhatnagar--Gross--Krook (BGK) operator.}
and a Multiple Relaxation Times (MRT) operator (for more details about these
operators, see the thesis of D. A. Beugre~\cite{Beugre10}). The MRT operator is
more stable numerically than the SRT operator, in particular when simulating
turbulent flows. However, it is much more complex to compute (a multiplication
by about 15 of execution time was observed for simulations using MRT instead of
SRT).

The solid (see section~\ref{sec_conf_lbConf_sol}) the fluid is flowing through is
represented using a bitmap i.e. a 3D array of same size as lattice indicating,
for each point, if it is an obstacle or not.

Pressure conditions (see section~\ref{sec_conf_proc_op}) are configured by
giving the pressure applied to the input face of the lattice and the pressure
applied on the output face of the lattice. These 2 faces are parallel, the flow
therefore follows one of the space axises.

The body force (see section~\ref{sec_conf_proc_op}) applied at each point of the
fluid is given by the components of the force along $x$, $y$ and $z$ axises.


%%%%%%%%%%%%%%%%%%%%%%%%%%%%%%%%%%%%%%%%%%%%%%%%%%%%%%%%%%%%%%%%%%%%%%%%%%%%%%%%
\section{Probes}
\label{sec_feat_prob}

Probes can be setup in order to regularly follow some properties of the flow
(see section~\ref{sec_conf_proc_log}). Available probes measure the speed and
density of fluid at given points or for complete slices of the lattice. New
probes may easily be developed in order to extend LaBoGrid's capabilities (see
section~\ref{sec_ext_log}).


%%%%%%%%%%%%%%%%%%%%%%%%%%%%%%%%%%%%%%%%%%%%%%%%%%%%%%%%%%%%%%%%%%%%%%%%%%%%%%%%
\section{Simulations workflow}
\label{sec_feat_work}

LaBoGrid is able to execute a sequence of LB flow simulations. The result of
each simulation can be stored to disk and/or be reused as starting point
for subsequent simulation. The result of a simulation can be reused later to
resume the simulation or to be post-processed. Section~\ref{sec_conf_exp}
explains how such workflows can be described.

Simulation workflows enable to automatically store the state of a simulation at
given points in time or to change some simulation parameters, like used probes,
during a simulation. For example, in a simulation lasting 10000 time steps,
the state of the flow is required at steps 5000 and 10000, and a probe must be
used from steps 5000 to 7000 in order to follow flow speed at some given points
of the lattice. The simulation can be desrcribed as a workflow of
3~simulations:

\begin{itemize}
\setlength{\itemsep}{0ex}
	\item time steps 0--5000: no probe is used, result is stored onto disk at the
	end of the simulation;
	\item time steps 5000--7000: a probe is used to regularly measure flow's
	speed at given points, result is not stored onto disk;
	\item time steps 7000--10000: no probe is used, result is stored onto disk at
	the end of the simulation;
\end{itemize}


%%%%%%%%%%%%%%%%%%%%%%%%%%%%%%%%%%%%%%%%%%%%%%%%%%%%%%%%%%%%%%%%%%%%%%%%%%%%%%%%
\section{Post-processing}

LaBoGrid itself does not provide post-processing capabilities but includes a
tool that allows to convert files resulting from a simulation into portable text
files whose content is easily exploitable (see section~\ref{sec_add_conv}).


%%%%%%%%%%%%%%%%%%%%%%%%%%%%%%%%%%%%%%%%%%%%%%%%%%%%%%%%%%%%%%%%%%%%%%%%%%%%%%%%
\section{Data handling server}
\label{sec_feat_stand}

LaBoGrid consumes and produces large amounts of data. Solid and/or result files
may be required for the execution of a simulation. In addition, the result of a
simulation may be stored. Finally, probes produce informations that should also
be stored in order to be processed.

When executing simulations on a single machine, all data are read from and
written to disk. When executing LaBoGrid on a cluster, input files must be
available for each computer of the cluster. In addition, it may be convinient
that all result files are centralized on a single computer the experimenter has
easy access to.

LaBoGrid therefore provides a ``stand alone server'' (see
section~\ref{sec_add_sas}). This application is executed on a computer that
will:

\begin{itemize}
\setlength{\itemsep}{0ex}
	\item provide input files to cluster computers,
	\item retrieve result files from cluster computers,
	\item log measures received from probes.
\end{itemize}

