\chapter{Additional tools}
\label{sec_add}

%%%%%%%%%%%%%%%%%%%%%%%%%%%%%%%%%%%%%%%%%%%%%%%%%%%%%%%%%%%%%%%%%%%%%%%%%%%%%%%%
\section{Result files converter}
\label{sec_add_conv}

Result files produced by LaBoGrid contain binary data (serialized objects).
These data may be compressed. This format is not suitable if the content of
these files must be processed using tools other than Java. Therefore, LaBoGrid
provides a tool that converts the content of its binary result files
into ``portable'' text files. These files contain macro variables such as
density and speed evaluated at each point of the lattice.

The conversion tool produces several output files: 1 per slice of the lattice
(the user must choose if the tool will produce slices that are perpendicular to
$x$-, $y$- or $z$-axis). Each file has name ``$\alpha$Slice\_$\beta$.txt'' where
$\alpha$ is the slice type (xy, yz or xz) and $\beta$ the position of the slice.

Each file contains $M$ lines, $M$ being the number of sites in each slice (all
slices contain the same number of sites). A line has following format
%
\begin{center}
$s_x$ $s_y$ $s_z$ $\frac{d}{3}$ $\sigma$
\end{center}
%
where $s_i$ is the speed along $i$-axis, $d$ is the local density, $\sigma$ is a
flag indicating if the site is an obstacle ($\sigma=1$) or fluid ($\sigma=0$).

For example, let a lattice have size $(a, b, c)$:
\begin{itemize}
	\item with xy slice type, $c$ ``xySlice\_$i$.txt'' files are produced with $0
	\leq i < c$,
	\item with xz slice type, $b$ ``xzSlice\_$i$.txt'' files are produced with $0
	\leq i < b$,
	\item with yz slice type, $a$ ``yzSlice\_$i$.txt'' files are produced with $0
	\leq i < a$.
\end{itemize}

Following pseudo-code illustrates the way data from converted files should be
read for post-processing in the case xy slices where requested (other slice
types are handled in a similar way). \texttt{xSize}, \texttt{ySize} and
\texttt{zSize} are the size of the lattice along each axis.

\begin{Verbatim}[tabsize=2,frame=lines]
for(int z = 0; z < zSize; ++z) {
    "Open file xySlice_z.txt" ;
    for(int y = 0; y < ySize; ++y) {
        for(int x = 0; x < xSize; ++x) {
            "Parse next line of file (data associated to (x,y,z)" ;
            "Exploit parsed data" ;
        }
    }
    "Close file" ;
}
\end{Verbatim}


%%%%%%%%%%%%%%%%%%%%%%%%%%%%%%%%%%%%%%%%%%%%%%%%%%%%%%%%%%%%%%%%%%%%%%%%%%%%%%%%
\section{Stand alone server}
\label{sec_add_sas}

A Stand Alone Server (SAS) is instantiated on a computer in order to send input
files to workers and receive result files or log data from them (see
sections~\ref{sec_conf_exp_io} and~\ref{sec_conf_proc_log}).

Following class provides the ``main'' method running a SAS when invoked:
\begin{center}
\texttt{laboGrid.standalone.StandAloneDistributedAgent}
\end{center}

A SAS takes 3 arguments:
\begin{enumerate}
	\item the host name of the computer hosting it,
	\item the port the SAS should accept connections on,
	\item the path to a folder where logs and temporary files are stored (created
	if it does not exist).
\end{enumerate}
